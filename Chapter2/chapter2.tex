%!TEX root = ../thesis.tex
%*******************************************************************************
%****************************** Second Chapter *********************************
%*******************************************************************************

\chapter{Caracterización Sucesos de Eventos}

\ifpdf
    \graphicspath{{Chapter2/Figs/Raster/}{Chapter2/Figs/PDF/}{Chapter2/Figs/}}
\else
    \graphicspath{{Chapter2/Figs/Vector/}{Chapter2/Figs/}}
\fi

Como se mencionó en el capitulo anterior, los sistemas WfM/BPM almacenan todos los sucesos de eventos generados al procesar cualquier elemento de trabajo. Estos sucesos de eventos o event logs en inglés, pueden estar relacionados por ejemplo con la creación de un nuevo elemento de trabajo, la asignación de un determinado elemento de trabajo a un usuario específico, la llegada de un elemento de trabajo a un cola específica, la invocación a un sistema externo, entre otras.

Específicamente para nuestro caso de estudio, los sucesos de eventos son almacenados en una única tabla, la cual se encuentran definida dentro de una base de datos relacional. La cantidad de registros de esta tabla crece con relación al número de elementos de trabajos procesados por el sistema, lo que quiere decir que para procesos que gestionan grandes cantidades de elementos de trabajo, se tienen grandes volúmenes de información. Esta tabla de sucesos de eventos cuenta con varios campos importantes, los cuales proveen información relevante al momento de identificar el comportamiento de los elementos de trabajo.

En vista de lo anterior, y teniendo en cuenta que, el primer paso para abordar un problema haciendo uso de la inteligencia computacional es la caracterización de la base de datos, a continuación se describirá todo el proceso llevado a cabo para realizar dicha tarea.

%********************************** %First Section  **************************************
\section{Selección de Características} %Section - 2.1 
\label{section2.1}

Al hacer uso de técnicas de machine learning o de inteligencia computacional, es usual que no sea necesario tener en cuenta todas las características de la base de datos, pues generalmente con solo algunas de estas características es posible describir adecuadamente el comportamiento de los datos. Adicional a lo anterior, tener en cuenta todas las características de la base de datos podría incrementar el costo computacional, al igual que podría hacer que los resultados obtenidos no sean los mejores. 

Para nuestro caso, se consideraron las siguientes características, las cuales fueron seleccionadas a partir del conocimiento específico de la herramienta tecnológica que las genera.


%********************************** %Second Section  **************************************
\section{One Hot Encoding} %Section - 2.2 
\label{section2.2}

Como se puede ver en la sección anterior, todas las características seleccionadas son categóricas. Lo implica que sea necesario realizar algún tipo de transformación sobre los datos, de tal manera que sea posible modelar adecuadamente los datos, al igual que aplicar las técnicas de machine leaning propuesta en este trabajo. En este orden de ideas, el mecanismo usado para realizar la categorización de los sucesos de eventos es el llamado One Hot Encoding. 

La técnica de One Hot Encoding sugiere la creación de nuevas columnas sobre la base de datos, donde cada nueva columna corresponde a una categoría de alguna de las características que componen la base de datos. Además, su valor será uno si determinado registro se encuentra asociado a dicha categoría, y en caso contrarío sería cero. La figura tal ilustra de una mejor manera lo mencionado anteriormente.  

Este proceso de One Hot Encoding da como resultado para cada registro de la base de datos un vector binario, tal como se muestra en la figura tal.

%********************************** %Third Section  **************************************
\section{Métodos de cuantización vectorial evaluados} %Section - 2.3 
\label{section2.3}

Una vez realizada la categorización de los sucesos de eventos, es necesario convertir los vectores binarios mencionados en la sección anterior a números enteros, de tal manera que sea más fácil almacenarlos y manipularlos. La figura tal muestra un ejemplo de como lucen las secuencias de sucesos de eventos después de todo el proceso realizado anteriormente. 

Sin embargo, la estructura obtenida para las secuencias de sucesos de eventos mostrada en la figura tal presentan un inconveniente, pues los valores obtenidos para cada uno de los sucesos de eventos no conforman una secuencia continua. Es por esto que debe realizarse un proceso denominado cuantización vectorial sobre los nuevos datos obtenidos. Este proceso busca que los nuevos datos sean almacenados de la forma más compacta posible, y también que sean representados con suficiente precisión, de tal manera que sea posible generar una secuencia continua con dichas observaciones [51]. Agregar figura. 

Generalmente, este proceso de cuantización vectorial se lleva a cabo haciendo uso de técnicas o métodos de clasificación. En la literatura se ha encontrado que el método más usado es el K-means. Sin embargo, la condición de los datos que se están modelando en este trabajo suponen una dificultad, pues las observaciones obtenidas corresponden a vectores binarios, sobre los cuales no se puede aplicar una medida de distancia como la euclidiana, y por el contrario es necesario usar la distancia hamming. 

Debido a que tal vez los resultados obtenidos con la técnica de K-means y la distancia hamming no sean los mejores, se hace uso también de la técnica de K-mode, la cual toma en cuenta la moda de las distancias de los diferentes puntos. Esta técnica es comúnmente usada para clasificar secuencias genéticas.   

En conclusión, dentro del proceso de cuantización de los datos se tuvieron en cuenta tres métodos. El método de K-means, el método de K-moda, y por último, un método que no hace uso de técnicas de clasificación, pero que le asigna un único identificador a cada observación.


%llever a la seccion de entrenamiento e implementación
es necesario generar las secuencias de sucesos de eventos, con estas secuencias se podrán identificar patrones en el comportamiento de los work items. Esto es importante debido que se esta abordando un problema relacionado con aprendizaje estructura, es decir que existen una dependencia entre cada nuevo suceso de evento con el anterior. Adicional, esta sería la estructura que soportan los modelos que son implementados para este trabajo.

Varias tareas se deben tener en cuenta para generar dichas secuencias de sucesos, primero se deben separar las secuencias que están asociadas a un error con las secuencias que no lo están, luego a partir de la característica de tiempo se ordenan los registros y se genera la secuencias con orden temporal. Para nuestro caso se estudio, la longitud de las secuencias pueden variar bastante, puedes existir secuencias muy pequeñas, pero también secuencias muy extensas, el maximo de observaciones es hasta de seis mil.
La sigueinte figura ilustra el proceso de generación de las secuencias de sucesos. Figura
%hasta acá



%********************************** %Quarter Section  **************************************
\section{Métodos de cuantización vectorial evaluados} %Section - 2.4 
\label{section2.4}


