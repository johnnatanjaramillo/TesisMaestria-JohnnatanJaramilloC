%!TEX root = ../thesis.tex
%*******************************************************************************
%*********************************** Introduction ******************************
%*******************************************************************************

\chapter*{Introducción \markboth{Introduction}{}}

La gestión de procesos de negocio (BPM, Business Process Management) es una metodología que busca controlar, analizar y mejorar los procesos organizacionales mediante técnicas, métodos y herramientas de software. Estas herramientas de software permiten obtener múltiples abstracciones de los procesos organizacionales, mediante la definición de flujos de trabajo o workflows \cite{VanderAalst2004}. Cada instancia de estos workflows se denomina un elemento de trabajo o workitem, y generalmente para cada uno de estos se registra un gran número de sucesos de eventos, los cuales pueden ser almacenados en una base de datos o en un archivo plano. El objetivo de estas herramientas de software es controlar los procesos organizacionales o de negocio, al igual que mejorar su eficiencia.

Sin embargo, la implementación de procesos de negocio sobre herramientas BPM presenta varias dificultades, por ejemplo, se puede tener un mal diseño del proceso, un mal desempeño a la hora de procesar los elementos de trabajo activos en el sistema, inconsistencia en la información almacenada, entre otras. Adicional a esto, en los procesos de negocio, generalmente un error o falla puede darse debido a un comportamiento anómalo en el proceso, es decir, cuando se presenta una secuencia de sucesos de eventos no esperada. Estos comportamientos pueden darse a nivel de plataforma o de proceso de negocio, y podrían ser inferidos desde las secuencias de sucesos de eventos registradas por el sistema. 

Es por lo anterior que la minería de procesos surge como una alternativa técnica que permite resolver dichas dificultades. Esta hace uso de técnicas de aprendizaje automático, lo cual permite identificar realmente cómo trabajan las personas sobre los procesos de negocio \cite{VanDerAalst2004_2}, al igual describir el comportamiento de los procesos de negocio, rediseñarlos y mejorarlos continuamente \cite{VanderAalst2007}. Todo lo anterior es bastante relevante, debido al actual interés de predecir el comportamiento de los procesos de negocio, dado que estos cada vez son más complejos \cite{Pandey2011}. 

Las técnicas de modelado usadas por la minería de procesos, y la estructura con que los sucesos de eventos son almacenados en el sistema, frecuentemente se enfocan en identificar solamente comportamientos relacionados con el tiempo de completitud de los elementos de trabajo, o con los costos que esto implica, dejando de lado los elementos de trabajo que se encuentran en un estado de error o que podrían estarlo \cite{VanDerAalst2011}. Sin embargo, en la actualidad existe el interés de contar con modelos que permitan realizar predicciones más específicas, es decir, predicciones relacionadas con el comportamiento y los estados finales que presentarán los diferentes elementos de trabajos activos en el proceso de negocio \cite{Camara2015}. 

A partir de lo anterior, las técnicas usadas en la minería de procesos podrían ser extendidas de tal manera que sea posible monitorear el comportamiento de los procesos de negocio, teniendo en cuenta las secuencias de sucesos de eventos generadas por los sistemas BPM. Esto sería de mucha utilidad al momento de realizar predicciones de posibles errores \cite{Kang2014}, al igual que permitiría contar con sistemas tolerantes a comportamientos indeseados o anómalos, y que apliquen acciones correctivas automáticamente al presentarse este tipo de comportamiento, o por lo menos que notifique la probabilidad de su ocurrencia \cite{Salfner2007,Yu2006}.

Un aspecto importante de los procesos de negocio actuales es que presentan un alto grado de variabilidad y comportamientos muy específicos \cite{Ferreira2007}. Por tal motivo las técnicas convencionales no siempre son útiles para analizar procesos de carácter dinámico, y tampoco cuentan con métricas que permitan evaluar la calidad del modelo obtenido respecto a la realidad \cite{Rozinat2008}.

En la minería de procesos es común el uso de modelos probabilísticos debido a la flexibilidad que estos ofrecen \cite{DaSilva2009}. Modelos como el de estados ocultos de Markov (HMM, Hidden Markov Model) son frecuentemente usados debido a su capacidad para modelar información correspondiente a secuencias temporales \cite{DaSilva2009}. Los HMM se ajustan adecuadamente a datos históricos, al igual que a los cambios presentados en los procesos de negocio a lo largo del tiempo \cite{Rozinat2008}. Sin embargo, los HMMs estándar no tienen en cuenta el tiempo que los elementos de trabajo consumen en cada uno de los estados del proceso, lo que limita su capacidad a la hora de predecir comportamientos atípicos sobre elementos de trabajo que permanecer mucho tiempo en un determinado estado del proceso \cite{DaSilva2009}. Ante estos y otros inconvenientes, surge una variación de HMM, esta permite detectar secuencias específicas a partir del análisis de los sucesos de eventos registrados por el sistema, y tiene en cuenta su tiempo de ocurrencia. Este modelo es el modelo de estados ocultos de semi-Markov (HSMM, Hidden Semi-Markov Model).

En HSMM la duración de un estado es explícitamente definida y está dada por una variable aleatoria, lo que quiere decir que la probabilidad de duración de un estado es establecida por una función de distribución, que puede ser una función de densidad de probabilidad continua, tal como una distribución Gaussiana, de Poisson o Gamma \cite{Marhasev2006}. 

También existe una extensión de HSMM que modela la dependencia de las probabilidades de transición en la duración de los estados, llamado modelo de estados ocultos de semi-Markov no estacionario (NHSMM, Non-stationary Hidden Semi-Markov Model) \cite{Marhasev2006}. Este modelo toma en cuenta no sólo la duración explícita de un estado, sino que también selecciona la transición de acuerdo a la duración exacta, mejorando la precisión del modelo \cite{Marhasev2006}. 

En el contexto de la minería de procesos, este compartimiento cobra mucho sentido, debido a que dependiendo del tiempo que dura un ítem de trabajo en un estado específico, se puede inferir que se está presentando un comportamiento anómalo en dicho ítem de trabajo, y que podría finalizar en un estado de error. Sin embargo, NHSMM no ha sido ampliamente utilizado en la minería de procesos debido al alto costo computacional que presenta el entrenamiento del modelo.

Lo anterior no quiere decir que HMM y HSMM no presenten una complejidad computacional alta, de hecho en estos la complejidad computacional aumenta a medida que se adicionan parámetros al modelo, dicha complejidad puede aumentar de manera exponencial con respecto al número de secuencias de eventos analizados \cite{Maurer2014}. Sin embargo para HSMM y NHSMM el costo computacional es mucho mayor que HMM aún para conjuntos de datos reducidos \cite{Johnson2005}.

Otras técnicas de aprendizaje automático más recientes como las redes neuronales artificiales también han sido exploradas en el contexto de la minería de procesos, sin embargo estas no han alcanzado el mismo interés que los modelos mencionados anteriormente. Esto, tal vez debido al alto grado de abstracción que se tiene al momento de implementar dichos modelos. 

Este costo computacional, representa una limitante al momento de entrenar los modelos sobre cientos o miles de secuencias de observaciones con una gran longitud, las cuales son generadas actualmente por las herramientas de software que usan metodologías BPM o WFM. Debido a esto, la implementación de estrategias en la minería de procesos que se basen en modelos como HMM, HSMM o NHSMM requieren del desarrollo de metodologías eficientes, tanto para el procesamiento de la información como para el entrenamiento de los modelos. En este trabajo se propone el desarrollo de una metodología que permita realizar el entrenamiento de los modelos HMM, HSMM y NHSMM haciendo uso del paradigma Spark, permitiendo así la utilización de dichos modelos para la predicción de comportamientos anómalos a partir de las secuencias de eventos registradas por un sistema, y que se generan en un proceso de negocio dado, siendo almacenadas en una base de datos.

%%% ----------------------------------------------------------------------